\part*{Appendix}
\label{appendix}
\addcontentsline{toc}{part}{Appendix}

\appendix %---------------------------------------


\chapter{Setup}
\label{chap:app:setup}
Install opam\footnote{\url{http://opam.ocamlpro.com/}}, then do
\begin{lstlisting}[language=Bash]
opam install ocamlfind camomile batteries cil xml-light
\end{lstlisting}
to install the latest versions of the dependencies for the current user.

%Since the version of CIL in the repository is too old, it has to be installed separately:
%\begin{lstlisting}
%git clone git://cil.git.sourceforge.net/gitroot/cil/cil
%pushd cil
%./configure && make && sudo make install
%popd
%rm -rf cil
%\end{lstlisting}
After that you can build goblint:

\begin{lstlisting}[language=Bash]
git clone https://github.com/vogler/analyzer.git
cd analyzer
make
\end{lstlisting}
If something goes wrong, take a look at \verb|scripts/travis-ci.sh| for an example setup or try the versions listed in \verb|INSTALL|.

Alternatively you can use your system's package manager to install the dependencies globally or use \verb|scripts/install_script.sh| to build everything from source without affecting any existing OCaml installation.


\paragraph*{Virtual machine}
A ready-to-use virtual machine can be started and connected to using Vagrant \footnote{\url{http://www.vagrantup.com/}}:
\begin{lstlisting}[language=Bash]
vagrant up	# download and provision VM
vagrant ssh	# connect over ssh
sudo su -		# change to root
cd analyzer	# goblint is ready here
\end{lstlisting}


\paragraph*{Web frontend}
In order to setup the web frontend, make sure Node.js\footnote{\url{http://nodejs.org/}} is installed and do:
\begin{lstlisting}[language=Bash]
git submodule update --init --recursive	# fetch code
cd webapp
sudo npm install -g coffee-script nodemon bower	# install those globally if not already installed (nodemon and bower are optional)
npm install	# locally install node and bower dependencies
\end{lstlisting}
Then run it using \verb|coffee server.coffee| or \verb|nodemon server.coffee| for automatic reloading during development.

A JavaScript version can be compiled using \verb|coffee -c server.coffee|.


\chapter{Usage}
\label{chap:app:usage}
\section{Command-line options}
\begin{lstlisting}
Usage: goblint [options] source-files
Options
    -v                        Prints more status information.                 
    -o <file>                 Prints the output to file.                      
    -I <dir>                  Add include directory.                          
    -IK <dir>                 Add kernel include directory.                   

    --help                    Prints this text                                
    --version                 Print out current version information.          

    --conf <file>             Merge the configuration from the <file>.        
    --writeconf <file>        Write the effective configuration to <file>     
    --set <jpath> <jvalue>    Set a configuration variable <jpath> to the specified <jvalue>.
    --sets <jpath> <string>   Set a configuration variable <jpath> to the string.
    --enable  <jpath>         Set a configuration variable <jpath> to true.   
    --disable <jpath>         Set a configuration variable <jpath> to false.  

    --print_options           Print out commonly used configuration variables.
    --print_all_options       Print out all configuration variables.          

A <jvalue> is a string from the JSON language where single-quotes (') are used instead of double-quotes (").

A <jpath> is a path in a json structure. E.g. 'field.another_field[42]';
in addition to the normal syntax you can use 'field[+]' append to an array.
\end{lstlisting}


\section{Generating a control flow graph}
In order to generate a control flow graph for the program \verb|test.c|, do the following:
\begin{lstlisting}
./goblint --set justcfg true test.c
dot -Tpng cfg.dot -o cfg.png 
\end{lstlisting}


\section{Using the implemented analyses}
\label{sec:app:use_ana}
\subsection{File handles}
To use the file analysis on \verb|test.c| with HTML output in \verb|./result|:
\begin{lstlisting}
./goblint --sets ana.activated[0][+] file --sets result html test.c
\end{lstlisting}
The analysis can be configured to be optimistic about opening files. If this option is set to \verb|true|, calls to \verb|fopen| will be assumed to never fail:
\begin{lstlisting}
./goblint --sets ana.activated[0][+] file --set ana.file.optimistic true --sets result html test.c
\end{lstlisting}

\subsection{Specification}
To use the spec analysis on \verb|test.c| with specification file \verb|test.spec|:
\begin{lstlisting}
./goblint --sets ana.activated[0][+] spec --sets ana.spec.file test.spec --sets result html test.c
\end{lstlisting}
\paragraph*{Parser}
The specification parser can be built and tested independent of Goblint. The following script compiles the parser and runs it on specification file \verb|$spec|. On failure, it starts the parser again in REPL-mode for debugging.
\begin{lstlisting}[language=bash]
bin=src/mainspec.native
spec=${1-"src/spec/file.spec"}
ocamlbuild -yaccflag -v -X webapp -no-links -use-ocamlfind $bin \
    && (./_build/$bin $spec \
        || (echo "$spec failed, running interactive now...";
            rlwrap ./_build/$bin
           )
       )
\end{lstlisting}
\paragraph*{Graph}
The parser saves a graph representing the specification to \verb|result/graph.dot| by default (see \verb|src/spec/specUtil.ml|).
To generate an image, the following could be used:
\begin{lstlisting}
dot -Tpng result/graph.dot -o graph.png 
\end{lstlisting}
