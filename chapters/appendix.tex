\part*{Appendix}
\label{appendix}
\addcontentsline{toc}{part}{Appendix}

\appendix %---------------------------------------


\chapter{Setup}
\label{chap:app:setup}
Install opam\footnote{\url{http://opam.ocamlpro.com/}}, then do
\begin{lstlisting}
opam install ocamlfind camomile batteries cil xml-light
\end{lstlisting}
to install the latest versions of the dependencies for the current user. After that you can build goblint:

\begin{lstlisting}
git clone https://github.com/goblint/analyzer.git
cd analyzer
make
\end{lstlisting}
If something goes wrong, switch to the versions listed in \verb|INSTALL|:
\begin{lstlisting}
opam switch 4.00.1
opam install ocamlfind.1.3.3 camomile.0.8.3 batteries.2.0.0 cil.1.5.1 xml-light.2.2
\end{lstlisting}
Alternatively you can use your system's package manager to install the dependencies globally or use \verb|install_script.sh| to build everything from source without affecting any existing OCaml installation.

\paragraph*{Web frontend}
In order to setup the web frontend, make sure Node.js\footnote{\url{http://nodejs.org/}} is installed and do:
\begin{lstlisting}
git submodule update --init --recursive
cd webapp
sudo npm install -g bower coffee-script nodemon     # install those globally if not already installed (nodemon is optional)
npm install && bower install
\end{lstlisting}
Then run it using \verb|coffee server.coffee| or \verb|nodemon server.coffee| for automatic reloading during development.

A JavaScript version can be compiled using \verb|coffee -c server.coffee|.


\chapter{Usage}
\label{chap:app:usage}
\section{Command-line options}
\begin{lstlisting}
Usage: goblint [options] source-files
Options
    -v                        Prints more status information.                 
    -o <file>                 Prints the output to file.                      
    -I <dir>                  Add include directory.                          
    -IK <dir>                 Add kernel include directory.                   

    --help                    Prints this text                                
    --version                 Print out current version information.          

    --conf <file>             Merge the configuration from the <file>.        
    --writeconf <file>        Write the effective configuration to <file>     
    --set <jpath> <jvalue>    Set a configuration variable <jpath> to the specified <jvalue>.
    --sets <jpath> <string>   Set a configuration variable <jpath> to the string.
    --enable  <jpath>         Set a configuration variable <jpath> to true.   
    --disable <jpath>         Set a configuration variable <jpath> to false.  

    --print_options           Print out commonly used configuration variables.
    --print_all_options       Print out all configuration variables.          

A <jvalue> is a string from the JSON language where single-quotes (') are used instead of double-quotes (").

A <jpath> is a path in a json structure. E.g. 'field.another_field[42]';
in addition to the normal syntax you can use 'field[+]' append to an array.
\end{lstlisting}


\section{Generating a control flow graph}
In order to generate a control flow graph do the following:
\begin{lstlisting}
./goblint --cfg test.c
dot -Tpng cfg.dot -o cfg.png 
\end{lstlisting}


\section{Using the implemented analyses}
\subsection{File handles}
To use the file analysis on \verb|test.c| with HTML output in \verb|./result|:
\begin{lstlisting}
./goblint --sets ana.activated[0][+] file --sets result html test.c
\end{lstlisting}

\subsection{Specification}
To use the spec analysis on \verb|test.c| with specification file \verb|test.spec|:
\begin{lstlisting}
./goblint --sets ana.activated[0][+] spec --sets spec.file test.spec --sets result html test.c
\end{lstlisting}
\paragraph*{Parser}
The specification parser can be built and tested independent from Goblint. The following script compiles the parser and runs it on specification file \verb|$spec|. On failure, it starts the parser again in REPL-mode for testing.
\begin{lstlisting}[language=bash]
bin=src/mainspec.native
spec=${1-"src/spec/file.spec"}
ocamlbuild -yaccflag -v -X webapp -no-links -use-ocamlfind $bin \
    && (./_build/$bin $spec \
        || (echo "$spec failed, running interactive now...";
            rlwrap ./_build/$bin
           )
       )
\end{lstlisting}
\paragraph*{Graph}
The parser saves a graph representing the specification to \verb|./result/graph.dot| by default (see \verb|src/spec/specUtil.ml|).
To generate an image, the following could be used:
\begin{lstlisting}
dot -Tpng result/graph.dot -o graph.png 
\end{lstlisting}
