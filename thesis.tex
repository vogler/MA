\documentclass[
fontsize=11pt,paper=a4,
bibliography=totoc, listof=totoc,
headsepline,footsepline,
footinclude=false,BCOR=12mm,DIV=13]{scrbook}

\usepackage[english]{babel}
\usepackage[utf8]{inputenc}
\usepackage{amsmath}
\usepackage{graphicx}

\title{Verifying Regular Safety Properties of C Programs Using the Static Analyzer Goblint}
\author{Ralf Vogler}


\usepackage{xcolor}
\usepackage{listings}
%% http://stackoverflow.com/questions/741985/latex-source-code-listing-like-in-professional-books
\usepackage{courier}
\definecolor{light-gray}{gray}{0.95}
\lstset{
  % language=C,
  basicstyle=\small\sffamily,
%   basicstyle=\small\ttfamily,
  numbers=left,
  numberstyle=\tiny,
  frame=tb,
  columns=fullflexible,
  showstringspaces=false,
	backgroundcolor=\color{light-gray},
	linewidth=\linewidth,       % Zeilenbreite
	breaklines=true,             % Zeileumbruch
	breakatwhitespace=false, %Umbruch an Leerzeichen
  tabsize=2,
  extendedchars=true,
  xleftmargin=17pt,
  framexleftmargin=17pt,
  abovecaptionskip=7pt,
%   frameround=tttt,
}

\newcommand{\listingC}[2]
{
  \lstinputlisting[language=C++,caption={#2},label=lst:#1]{code/17-file/#1}
}

\newcommand{\listingML}[2]
{
  \lstinputlisting[language=ML,caption={#2},label=lst:#1]{code/#1}
}

\begin{document}
\maketitle
\tableofcontents

%\begin{abstract}
%In this thesis a specification language is presented which can be used to verify regular safety properties of C programs. The work is based on the static analyzer Goblint as a framework. Verification of file handle usage serves as an example for comparing a manual implementation with the developed specification language. Finally other possible use cases and limitations are explored.
%\end{abstract}

% https://www.sharelatex.com/project/517466a40a231bd40a6e9f5c

\chapter{Introduction}



\chapter{Static analysis and Goblint}
\section{Overview}

\section{Theory}

\section{Getting started}
Appendix: Setup
Adding an own analysis


\chapter{Verifying correct usage of file handles}
\section{Common problems using files}
The following examples for common problems served as a guideline for the implementation and contain comments that indicate where warnings would be output.

Listing \ref{lst:01-ok.c} shows opening a file \verb|test.txt| and appending the line "Testing..." to it. At the end the file is closed.
\listingC{01-ok.c}{Everything fine?}
This seems fine, but what happens if the file can't be written to?
We forgot to check the result of \verb|fopen| which returns an error code if the file couldn't be opened successfully.
This case is handled by the implementation and can also be checked using the specification language. In this example warnings would be issued that the file handle might not be open after line 5.

For brevity reasons a success check is omitted in the following examples and it's assumed that the file could be opened without errors. The specification version can be easily adjusted to conform to this by just replacing a few states, whereas the manual implementation would always warn about maybe unopened file handles.
\\\\
Not closing files is not necessarily an error since file handles are closed at the end of the program anyway, but it's not good practice and might lead to unwanted program behavior.
Imagine a program that is done with a file but doesn't close it. If the program gets stuck in calculations or on user input, the file will still be locked and can't be used by other programs.

Listing \ref{lst:03-if-close.c} shows where warnings would be issued. There is a warning where the file was opened and a summary of unclosed files at the end of the program.
\listingC{03-if-close.c}{Missing fclose}

Listing \ref{lst:04-no-open.c} shows what happens if the file handle was not opened before using it. When running the program, this leads to a segmentation fault.
\listingC{04-no-open.c}{Missing fopen}

Writing to a file which is opened read-only as shown in Listing \ref{lst:05-open-mode.c} is another problem. Bugs of this kind might be hard to find, since this executes without errors - there is just nothing written to the file.
\listingC{05-open-mode.c}{Wrong open mode}



\section{A domain for representing file handle usage}
Must/May []

\section{An analysis for checking file handle usage}

\section{Soundness vs. Precision}


\chapter{A general specification for regular safety properties}
\section{Interesting types of constraints}

\section{Representing the state of properties using an automaton}

\section{Specification format}

\section{Shortcuts for making the specification more concise}
More sane/human-readable.


\chapter{Example use cases}
\section{File handles redux}

\section{Locks}
Different kinds of locks + table of functions for them.
Locks with counters not regular -> see extenstions.

\section{Heap usage: malloc and free}


\chapter{Providing a better interface}
\section{Web frontend}
Screenshots.
Online version.


\chapter{Tests and real world examples}
Test some specifications on kernel code?
Benchmarks?


\chapter{Ideas for further development}
\section{Non-regular safety properties}
E.g. locks with counters.


\chapter{Conclusion}




\section{Some \LaTeX{} Examples}
\label{sec:examples}

\subsection{Sections}

Use \texttt{section}s and \texttt{subsection}s to organize your document. \LaTeX{} handles all the formatting and numbering automatically. Use \texttt{ref} and \texttt{label} for cross-references --- this is Section~\ref{sec:examples}, for example.

\subsection{Tables and Figures}

Use \texttt{tabular} for basic tables --- see Table~\ref{tab:widgets}, for example. You can upload a figure (JPEG, PNG or PDF) using the files menu. To include it in your document, use the \texttt{includegraphics} command (see the comment below in the source code).

% Commands to include a figure:
%\begin{figure}
%\includegraphics[width=\textwidth]{your-figure's-file-name}
%\caption{\label{fig:your-figure}Caption goes here.}
%\end{figure}

\begin{table}
\centering
\begin{tabular}{l|r}
Item & Quantity \\\hline
Widgets & 42 \\
Gadgets & 13
\end{tabular}
\caption{\label{tab:widgets}An example table.}
\end{table}

\subsection{Mathematics}

\LaTeX{} is great at typesetting mathematics. Let $X_1, X_2, \ldots, X_n$ be a sequence of independent and identically distributed random variables with $\text{E}[X_i] = \mu$ and $\text{Var}[X_i] = \sigma^2 < \infty$, and let
$$S_n = \frac{X_1 + X_2 + \cdots + X_n}{n}
      = \frac{1}{n}\sum_{i}^{n} X_i$$
denote their mean. Then as $n$ approaches infinity, the random variables $\sqrt{n}(S_n - \mu)$ converge in distribution to a normal $\mathcal{N}(0, \sigma^2)$.

\subsection{Lists}

You can make lists with automatic numbering \dots

\begin{enumerate}
\item Like this,
\item and like this.
\end{enumerate}
\dots or bullet points \dots
\begin{itemize}
\item Like this,
\item and like this.
\end{itemize}

\end{document}
