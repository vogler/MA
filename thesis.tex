\documentclass[12pt]{scrreprt}

\usepackage[english]{babel}
\usepackage[utf8]{inputenc}
\usepackage{amsmath}
\usepackage{graphicx}

\title{Verifying Regular Safety Properties of C Programs Using the Static Analyzer Goblint}
\author{Ralf Vogler}

\begin{document}
\maketitle
\tableofcontents

\begin{abstract}
Your abstract.
\end{abstract}

% https://www.sharelatex.com/project/517466a40a231bd40a6e9f5c

\chapter{Introduction}


\chapter{Static analysis and Goblint}
\section{Overview}

\section{Theory}

\section{Getting started}
Appendix: Setup
Adding an own analysis


\chapter{Verifying correct usage of file handles}
\section{Common problems using files}
Examples/tests

\section{A domain for representing file handle usage}
Must/May []

\section{An analysis for checking file handle usage}

\section{Soundness vs. Precision}


\chapter{A general specification for regular safety properties}
\section{Interesting types of constraints}

\section{Representing the state of properties using an automaton}

\section{Specification format}

\section{Shortcuts for making the specification more concise}
More sane/human-readable.


\chapter{Example use cases}
\section{File handles redux}

\section{Locks}
Different kinds of locks + table of functions for them.
Locks with counters not regular -> see extenstions.

\section{Heap usage: malloc and free}


\chapter{Providing a better interface}
\section{Web frontend}
Screenshots.
Online version.


\chapter{Tests and real world examples}
Test some specifications on kernel code?
Benchmarks?


\chapter{Ideas for further development}
\section{Non-regular safety properties}
E.g. locks with counters.


\chapter{Conclusion}




\section{Some \LaTeX{} Examples}
\label{sec:examples}

\subsection{Sections}

Use \texttt{section}s and \texttt{subsection}s to organize your document. \LaTeX{} handles all the formatting and numbering automatically. Use \texttt{ref} and \texttt{label} for cross-references --- this is Section~\ref{sec:examples}, for example.

\subsection{Tables and Figures}

Use \texttt{tabular} for basic tables --- see Table~\ref{tab:widgets}, for example. You can upload a figure (JPEG, PNG or PDF) using the files menu. To include it in your document, use the \texttt{includegraphics} command (see the comment below in the source code).

% Commands to include a figure:
%\begin{figure}
%\includegraphics[width=\textwidth]{your-figure's-file-name}
%\caption{\label{fig:your-figure}Caption goes here.}
%\end{figure}

\begin{table}
\centering
\begin{tabular}{l|r}
Item & Quantity \\\hline
Widgets & 42 \\
Gadgets & 13
\end{tabular}
\caption{\label{tab:widgets}An example table.}
\end{table}

\subsection{Mathematics}

\LaTeX{} is great at typesetting mathematics. Let $X_1, X_2, \ldots, X_n$ be a sequence of independent and identically distributed random variables with $\text{E}[X_i] = \mu$ and $\text{Var}[X_i] = \sigma^2 < \infty$, and let
$$S_n = \frac{X_1 + X_2 + \cdots + X_n}{n}
      = \frac{1}{n}\sum_{i}^{n} X_i$$
denote their mean. Then as $n$ approaches infinity, the random variables $\sqrt{n}(S_n - \mu)$ converge in distribution to a normal $\mathcal{N}(0, \sigma^2)$.

\subsection{Lists}

You can make lists with automatic numbering \dots

\begin{enumerate}
\item Like this,
\item and like this.
\end{enumerate}
\dots or bullet points \dots
\begin{itemize}
\item Like this,
\item and like this.
\end{itemize}

\end{document}
