\renewcommand{\sectfont}{\normalfont \bfseries}        % Schriftart der Kopfzeile

% manipulate footer
\usepackage{scrpage2}
\pagestyle{scrheadings}
\ifoot[\footertext]{\footertext} % \footertext set in INFO.TEX
%\setkomafont{pagehead}{\normalfont\rmfamily}
\setkomafont{pagenumber}{\normalfont\rmfamily}

%% allow sophisticated control structures
\usepackage{ifthen}

% use Palatino as default font
\usepackage{palatino}

% enable special PostScript fonts
\usepackage{pifont}

% make thumbnails
% \usepackage{thumbpdf}

%to use the subfigures
%\usepackage{subfigure} % XX


\usepackage{colortbl}


%% show program code\ldots
% \usepackage{verbatim}
% \usepackage{program}

%% enable TUM symbols on title page
\usepackage{styles/tumlogo}


%\usepackage{multirow} % XX

%% use colors
\usepackage{color}

%% make fancy math
\usepackage{amsmath}
\usepackage{amsfonts}
\usepackage{amssymb}
\usepackage{textcomp}
%\usepackage{yhmath} % für die adots 
%% mark text as preliminary
%\usepackage[draft,german,scrtime]{prelim2e}

%% create an index
\usepackage{makeidx}

% for the program environment
\usepackage{float}

%% load german babel package for german abstract
%\usepackage[german,american]{babel}
% \usepackage[ngerman,english]{babel}
% \selectlanguage{english}

% use german characters as well
% \usepackage[latin1]{inputenc}       % allow Latin1 characters
\usepackage[utf8]{inputenc}

% use initals dropped caps - doesn't work with PDF
% \usepackage{dropping}


% \usepackage{styles/shortoverview}
%----------------------------------------------------
%      Graphics and Hyperlinks
%----------------------------------------------------
%% pdfTex
%% reduce output size \pdfcompresslevel=9
%% declare pdfinfo
%\pdfinfo { 
%  /Title (my title) 
%  /Creator (pdfLaTeX) 
%  /Author (my name) 
%  /Subject (my subject	) 
%  /Keywords (my keywords)
%}
%% use pdf or jpg graphics
\usepackage[pdftex]{graphicx}
\DeclareGraphicsExtensions{.jpg,.JPG,.png,.pdf,.eps}
\graphicspath{{graphics/}} 

%% Load float package, for enabling floating extensions
% \usepackage{float}

%% allow rotations
\usepackage{rotating}
%% use pdftex version of hyperref
\usepackage[pdftex,colorlinks=true,linkcolor=red,citecolor=blue,%
anchorcolor=red,urlcolor=red,bookmarks=true,%
bookmarksopen=true,bookmarksopenlevel=1,plainpages=false,%
bookmarksnumbered=true,hyperindex=false,pdfstartview=%
]{hyperref}
%\usepackage[hyphenbreaks]{breakurl}



%% Fancy chapters
%\usepackage[Lenny]{fncychap}
%\usepackage[Glenn]{fncychap}
%\usepackage[Bjarne]{fncychap}

%\usepackage[avantgarde]{quotchap}

% set the bibliography style
%\bibliographystyle{styles/bauermaNum}
%\bibliographystyle{alpha}
\bibliographystyle{plain}


%%%%% listings %%%%%%

%% grey captions
\usepackage{xcolor}
% \usepackage{caption}
% \DeclareCaptionFont{white}{\color{white}}
% \DeclareCaptionFormat{listing}{\colorbox{gray}{\parbox{\textwidth}{#1#2#3}}}
% \captionsetup[lstlisting]{format=listing,labelfont=white,textfont=white}

% \usepackage{scrhack}
\usepackage{listings}

%% Wiesi
% \lstset{
% %   language=C++,
% %   keywordstyle=\bfseries,
% %   emphstyle=\underbar,
%   captionpos=b,
% %   numbers=right,
% %   stepnumber=5,
% %   numberstyle=\small,
% %   xleftmargin=2em,
% %   xrightmargin=3em,
% %   frame=r
% }

%% http://stackoverflow.com/questions/741985/latex-source-code-listing-like-in-professional-books
\usepackage[]{sourcecodepro}	% Adobe Source Code Pro, included in texlive-fontsextra
\definecolor{light-gray}{gray}{0.95}
\definecolor{myblue}{rgb}{0,0,0.8}
\definecolor{mygreen}{rgb}{0,0.4,0}
\definecolor{mymauve}{rgb}{0.5,0,0.8}
\definecolor{mygray}{rgb}{0.5,0.5,0.5}
\lstset{
%  language=C,
  basicstyle=\small\sffamily,
%  basicstyle=\small\ttfamily,
  keywordstyle=\color{myblue},
  commentstyle=\color{mygreen}, 
  stringstyle=\color{mymauve},
  numberstyle=\tiny\color{mygray},
  numbers=left,
  captionpos=b,
  abovecaptionskip=7pt,
%  columns=fullflexible,
  showstringspaces=false,
  backgroundcolor=\color{light-gray},
  linewidth=\linewidth,		% Zeilenbreite
  breaklines=true,			% Zeileumbruch
  breakatwhitespace=false,	% Umbruch an Leerzeichen
  tabsize=2,
  extendedchars=true,
  xleftmargin=17pt,
  framexleftmargin=17pt,
  frame=tb,
%  frame=single,
%  frameround=tttt,
}


%% compact lists
\usepackage{mdwlist}